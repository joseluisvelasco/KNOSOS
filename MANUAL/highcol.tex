\chapter{High collisionalities}\label{CHAP_HIGHCOL}

In this appendix we describe how the code solves transport at higher collisionalities (for which the bounce-average technique cannot be applied). For bulk species, we will recover the results of \DKES~(i.e., without momentum conservation), while for trace impurities we will obtain those of~\citep{calvo2018nf}\footnote{We note that these sections have not been part of any peer-reviewed paper, and it is not impossible that some of the intermediate equations contain typos}.

%%%%%%%%%%%%%%%%%%%%%%%%%%%%%%%%%%%%%%%%%%%%%%%%%%%%%%%%%%%%%%%%%%%%%%%%%%%%%%%%%%%%

\section{Bulk species in the Pfisrch-Schl\"uter regime}

We repeat here the derivation of the appendix of~\citep{igitkhanov2006impurity}.  We want to calculate $h$, the deviation of the distribution function from the Maxwellian, and we assume $\varphi_1=0$. The equation that we want to solve is
\begin{equation}
\left(vp\frac{\bB}{B}+\frac{\bE\times\bB}{B^2}\right)\cdot\bnabla h + \bv_M\cdot\bnabla\psi \Upsilon F_M = \frac{\nu}{2}\frac{\partial}{\partial p}\left((1-p^2)\frac{\partial h}{\partial p}\right)\,.\label{EQ_DKEPS}
\end{equation}
where we have used a simplified collision operator that is not correct at these collisionalities. The variables in velocity space are the particle speed $v$ and the pitch-angle $p=v_\parallel/v$. Finally, we note that, for very high collisionalities, we have to keep the $\bE\times\bB$ drift. 

Using the expressions derived in \S\ref{SEC_USEFUL} for right-handed Boozzer coordinates, equation~(\ref{EQ_DKEPS})  becomes
\begin{eqnarray}
 \left[vp\frac{B}{B_\zeta+\iota B_\theta}\left(\iota\frac{\partial}{\partial \theta}+\frac{\partial}{\partial \zeta}\right)+ \frac{1}{|B_\zeta+\iota B_\theta|}\frac{\partial \varphi}{\partial\psi}\left(B_\zeta\frac{\partial}{\partial \theta}-B_\theta\frac{\partial}{\partial \zeta}\right)\right]  h +\nonumber\\ +v_d\frac{1+p^2}{2}\frac{1}{B|B_\zeta+\iota B_\theta|}\left(B_\theta\frac{\partial B}{\partial \zeta}-B_\zeta\frac{\partial B}{\partial \theta}\right) \Upsilon F_M = \frac{\nu}{2}\frac{\partial}{\partial p}\left((1-p^2)\frac{\partial h}{\partial p}\right)\,.
\end{eqnarray}
We now expand $h$ (and $B$) in Fourier components
\begin{equation}
h(\theta,\zeta,p) = \Upsilon F_M \sum_{m,n} h_{mn}(p)\exp{[i(m\theta+nN_p\zeta)]}\,,
\end{equation}
and $h_{mn}(p)$ in Legendre polynomia:
\begin{eqnarray}
h_{mn}(p) = \sum_l h_{mnl} P_l(p) = h_{mn0} + p h_{mn1} + \frac{1}{2}(3p^2-1) h_{mn2}\,.
\end{eqnarray}
We keep only the two first polynomia since it is easy to show that
\begin{eqnarray}
\int_{-1}^{+1}\mathrm{d}p (1+p^2) P_2(p) = 0.1 \int_{-1}^{+1}\mathrm{d}p (1+p^2) P_0(p)\,.
\end{eqnarray}
and, at large $\nu$, $h_{mn2}\ll h_{mn0}$. We now invoke the large aspect-ratio approximation and consider variations of the magnetic field strenght to be small compared to its average. In this limit
\begin{eqnarray}
\frac{B}{B_\zeta+\iota B_\theta} &\approx& \frac{\Psi_t'}{R}\,,\nonumber\\
\frac{1}{B} &\approx& \frac{1}{\overline{B}}\,,
\end{eqnarray}
where $R$ is the major radius of the device and $\overline{B}$ some average of the magnetic field strength\footnote{We take $\overline{B}\equiv B_{00}$.}. The drift-kinetic equation becomes, for each Fourier component
\begin{eqnarray}
i\left[vp\frac{\Psi_t'}{R}(\iota m  + nN_p)+\frac{1}{|B_\zeta+\iota B_\theta|}\frac{\partial \varphi}{\partial\psi}(m B_\zeta-nN_p B_\theta)\right] \left(h_{mn0} + p h_{mn1}+ \frac{1}{2}(3p^2-1) h_{mn2}\right) -\nonumber\\ +iv_d\frac{1+p^2}{2} \frac{1}{\overline{B}|B_\zeta+\iota B_\theta|} (n N_p B_\theta - m B_\zeta) B_{mn}  = \frac{\nu}{2} \left( -2 p h_{mn1} - 3(3p^2-1) h_{mn2}\right)\,.
\end{eqnarray}
and we have to solve a system of three equations
\begin{eqnarray}
i\omega_{mn}h_{mn0} + \left(-\frac{3}{2}\nu -i\frac{1}{2}\omega_{mn}\right)h_{mn2} = iv_{d,mn}\,,\nonumber\\
iu_{mn}h_{mn0} + (i\omega_{mn}+\nu) h_{mn1} = 0 \,,\nonumber\\
iu_{mn}h_{mn1}+\left(\frac{9}{2}\nu+i\frac{3}{2}\omega_{mn}\right)h_{mn2} = iv_{d,mn}\,,
\end{eqnarray}
where
\begin{eqnarray}
u_{mn} &=& v\frac{\Psi_t'}{R}(\iota m  + nN_p)\,,\nonumber\\
\omega_{mn} &=& \frac{\partial \varphi}{\partial\psi}(m B_\zeta-nN_p B_\theta)\,,\nonumber\\
v_{d,mn} & = &\frac{v_d}{2} \frac{1}{\overline{B}|B_\zeta+\iota B_\theta|} (m B_\zeta-n N_p B_\theta) B_{mn}  \,,
\end{eqnarray}
These equations can be combined into two:
\begin{eqnarray}
iu_{mn}h_{mn1} +i3\omega_{mn}h_{mn0} &=& i4v_{d,mn}\,,\nonumber\\
iu_{mn}h_{mn0} + (i\omega_{mn}+\nu) h_{mn1} &=& 0 \,,
\end{eqnarray}
and we can solve for $h_{mn0}$, the only one that contributes the the radial flux:
\begin{eqnarray}
h_{mn0} =i\frac{4v_{d,mn}u_{mn}^2\nu}{(u_{mn}^2-3\omega_{mn}^2)^2+(3\nu\omega_{mn})^2}\approx i\frac{4v_{d,mn}\nu}{u_{mn}^2\left[1+\left(\frac{3\nu\omega_{mn}}{u_{mn}^2}\right)^2\right]}\,,
\end{eqnarray}
where we have only kept the stellarator-asymmetric part and we have used that $u_{mn}^2\gg 3\omega_{mn}^2$. We can now calculate the neoclassical transport coefficient $D_{11}$ (note that we drop the species index):
\begin{equation}
D_{11}= -\frac{1}{2}\left(\frac{\partial r}{\partial\psi}\right)^2\fsa{\int_{-1}^{+1}\mathrm{d}p \frac{h}{\Upsilon F_M} \bv_M\cdot\bnabla\psi}\,,
\end{equation}
where the brackets denote flux-surface-average. Thanks to the large aspect-ratio approximation, we can take the Jacobian to be approximately constant and obtain 
\begin{eqnarray}
D_{11} &\approx& -\frac{1}{2}\left(\frac{\partial r}{\partial\psi}\right)^2\frac{1}{4\pi^2}\int_0^{2\pi}\mathrm{d}\theta\int_0^{2\pi}\mathrm{d}\zeta\int_{-1}^{+1}\mathrm{d}p \frac{h}{\Upsilon F_M} \bv_M\cdot\bnabla \psi = \nonumber \\ &=& -\frac{1}{2}\left(\frac{\partial r}{\partial\psi}\right)^2 \sum_{m,n}\int_{-1}^{+1}\mathrm{d}\,p \frac{h_{mn}}{\Upsilon F_M}(\bv_M\cdot\bnabla\psi)_{-m,-n}  = \sum_{m,n}(D_{11})_{mn}\,,
\end{eqnarray}
with
\begin{equation}
(D_{11})_{mn} \equiv -\frac{1}{2}\left(\frac{\partial r}{\partial\psi}\right)^2\int_{-1}^{+1}\mathrm{d}p\,\frac{h_{mn}}{\Upsilon F_M}(\bv_M\cdot\bnabla\psi)_{-m,-n}\,,
\end{equation}
and
\begin{equation}
 (\bv_M\cdot\bnabla\psi)_{mn} = -i(1+p^2)v_{d,mn}\,.
\end{equation}
Our expressions can be rewritten as
\begin{eqnarray}
(D_{11})_{mn}= -\frac{16}{3}\left(\frac{\partial r}{\partial\psi}\right)^2\frac{v_{d,mn}v_{d,-m,-n}\nu}{u_{mn}^2\left[1+\left(\frac{3\nu\omega_{mn}}{u_{mn}^2}\right)^2\right]}\,.\label{EQ_D11PS}
\end{eqnarray}

%%%%%%%%%%%%%%%%%%%%%%%%%%%%%%%%%%%%%%%%%%%%%%%%%%%%%%%%%%%%%%%%%%%%%%%%%%%%%%%%%%%%%


\section{Bulk species in the plateau regime}\label{SEC_PLATEAU}

We repeat here the derivation of~\citep{alonso2017JPP}, and expand it to include the effect of the radial $E\times B$ drift\footnote{Additionally, we note that the derivation of the plateau in~\citep{alonso2017JPP} contains a typo that we comment on later in this section.}. We want to calculate $h$, the deviation of the distribution function from the Maxwellian, described by the next equation:
 \begin{equation}
 vp\frac{\bB}{B}\cdot\bnabla h + \nu h = -(\bv_M+\bv_E)\cdot\bnabla\psi \Upsilon F_M\,.\label{EQ_DKEPLATEAU}
 \end{equation}
 The variables in velocity space are again the particle speed $v$ and the pitch-angle $p=v_\parallel/v$. In Boozer coordinates,  equation~(\ref{EQ_DKEPLATEAU}) becomes, in the large aspect-ratio approximation:
 \begin{eqnarray}
& &vp\frac{\Psi_t'}{R}\left(\iota\frac{\partial}{\partial \theta}+\frac{\partial}{\partial \zeta}\right) h + \nu h = \Upsilon F_M\frac{1}{|B_\zeta+\iota B_\theta|} \nonumber\\  &\times& \left[v_d\frac{1+p^2}{2\overline{B}}\left(B_\zeta\frac{\partial B}{\partial \theta}-B_\theta\frac{\partial B}{\partial \zeta}\right)+\left(B_\zeta\frac{\partial \varphi_1}{\partial \theta}-B_\theta\frac{\partial \varphi_1}{\partial \zeta}\right)\right]\,,
 \end{eqnarray}
 If we now expand $h$, $B$ and 
 \begin{equation}
 \varphi_1(\theta,\zeta) =  \sum_{m,n} \varphi_{1,mn}\exp{[i(m\theta+nN_p\zeta)]}\,,
 \end{equation}
 in Fourier components, we have, for each component
 \begin{equation}
vp\frac{\Psi_t'}{R}i(\iota m + nN_p) h_{mn} + \nu h_{mn} = \Upsilon F_M S_{mn}\,,
 \end{equation}
 with
  \begin{equation}
S_{mn}=   i\frac{(mB_\zeta -nN_pB_\theta)}{|B_\zeta+\iota B_\theta|}\left(v_d\frac{1+p^2}{2|B|}B_{mn}+\varphi_{1,mn}\right)\,.
 \end{equation}
We arrive to
 \begin{eqnarray}
 \frac{h_{mn}}{\Upsilon F_M} &=& \frac{S_{mn}}{\frac{\Psi_t'vp}{R}i(\iota m + nN_p)+\nu} = \frac{S_{mn}\left(\nu-i\frac{\Psi_t'vp}{R}(\iota m + nN_p)\right)}{\left(\frac{vp}{R}\right)^2(\iota m + nN_p)^2+\nu^2}\,.
 \end{eqnarray}
 Since we are going to calculate the particle and energy flux, we drop the term which is odd in $p$:
 \begin{eqnarray}
 \frac{h_{mn}}{\Upsilon F_M}  = \frac{S_{mn}\nu}{\left(\frac{vp}{R}\right)^2(\iota m + nN_p)^2+\nu^2}\,.
 \end{eqnarray}
 We can rearrange this expression
 \begin{equation}
 \frac{h_{mn}}{\Upsilon F_M}  = \frac{R}{v|\iota m + nN_p|}S_{mn}\frac{\nu^*}{p^2+(\nu^*)^2}\,,
  \end{equation}
 where $\nu^*=\frac{R\nu}{v|\iota m +nN_p|}$. Since we are interested in the lowest order (i.e. $O(\nu^0)$) contribution to the flux~\citep{alonso2017JPP}, we can use the fact that
 \begin{equation}
 \pi \delta(p)=\lim_{\nu^*\rightarrow 0}\frac{\nu^*}{p^2+(\nu^*)^2}\,,
 \end{equation}
 in order to integrate in the pitch angle and obtain the final expression:
 \begin{eqnarray}
  (D_{11})_{mn}&=& \frac{\pi R}{2v|\iota m + nN_p|}\left(\frac{\partial r}{\partial\psi}\right)^2 S_{m,n}S_{-m,-n} \int_{-1}^{+1}\mathrm{d}p\,(1+p^2)\frac{\nu^*}{p^2+(\nu^*)^2} = \nonumber\\ &=&  \frac{\pi R}{2v|\iota m + nN_p|}\left(\frac{\partial r}{\partial\psi}\right)^2 S_{m,n}(p=0)S_{-m,-n}(p=0)\,.\label{EQ_D11PLATEAU}
 \end{eqnarray}
 The cuasineutrality equation reads\footnote{We note that there is a typo in equation (3.56) of~\citep{alonso2017JPP} that leads to equation (3.58) predicting an incorrect value of $\varphi_1$, twice that of equation~\ref{EQ_QNPLATEAU}}
 \begin{eqnarray}
\left(\frac{Z_i}{T_i}+\frac{1}{T_e}\right) e \varphi_1&=& n_i \int_0^\infty\mathrm{d}v v^2 \int_{-1}^{+1}\mathrm{d}p\,h \nonumber\\ 
&=& n_i R \int_0^\infty\mathrm{d}v v \Upsilon F_M \sum_{n,m} \frac{S_{mn}}{|\iota m + nN_p|}\,.\label{EQ_QNPLATEAU}
 \end{eqnarray}


%& \frac{\pi R}{2v|\iota m + nN_p|}\left(\frac{\partial r}{\partial\psi}\right)^2 S_{m,n}S_{-m,-n} \int_{-1}^{+1}\mathrm{d}p\,(1+p^2)\frac{\nu^*}{p^2+(\nu^*)^2} = \nonumber\\ &=&  \frac{\pi R}{2v|\iota m + nN_p|}\left(\frac{\partial r}{\partial\psi}\right)^2 S_{m,n}(p=0)S_{-m,-n}(p=0)\label{EQ_D11PLATEAU}
% (D_{11})_{mn}&=& \frac{1}{8}\left(\frac{\partial r}{\partial\psi}\right)^2 \frac{v_d^2}{v}\frac{R}{\overline{B}^2|B_\zeta+\iota B_\theta|^2}\frac{(mB_\zeta-nN_pB_\theta)^2B_{mn}B_{-m,-n}}{|\iota m+nN_p|}\int_{-1}^{+1}\mathrm{d}p\,(1+p^2)\frac{\nu^*}{p^2+(\nu^*)^2} = \nonumber\\ &=& \frac{\pi}{8} \frac{v_d^2}{v}\frac{R}{\overline{B}^2|B_\zeta+\iota B_\theta|^2}\frac{(mB_\zeta-nN_pB_\theta)^2B_{mn}B_{-m,-n}}{|\iota m+nN_p|}\,.\label{EQ_D11PLATEAU}

%%%%%%%%%%%%%%%%%%%%%%%%%%%%%%%%%%%%%%%%%%%%%%%%%%%%%%%%%%%%%%%%%%%%%%%%%%%%%%%%%%%%

\section{Classical flux and neoclassical flux of trace impurities species in the presence of bulk ions at low collisionalities}\label{SEC_CLANC}.

Here we follow~\citep{calvo2018nf,calvo2019anis}.


