\chapter{Concluding remarks}\label{CHAP_CONC}

%%%%%%%%%%%%%%%%%%%%%%%%%%%%%%%%%%%%%%%%%%%%%%%%%%%%%%%%%%%%%%%%%%%%%%%%%%%%%%%%%%%%%%%%%%%%%%%%%%%%%%%%%%%%%%%%%%%

\footnote{This chapter corresponds to section 5 of~\citep{velasco2019knosos}.}
{\ttfamily KNOSOS} is a freely-available open-source code that provides a fast computation of neoclassical transport at low collisionality in three-dimensional magnetic confinement devices, thanks to a rigorous application of the orbit-averaging technique to the drift-kinetic equation and an efficient solution of the quasineutrality equation. We have shown that, when solving equivalent equations, {\ttfamily KNOSOS} reproduces the calculations of~\DKES~and {\ttfamily EUTERPE} in simulations that can be orders of magnitude faster. This makes it a tool that can be used for a variety of physics problems, that we summarize next.

As a first obvious application, it can provide a fast calculation of the level of transport of a magnetic configuration for low-collisionality transport regimes not usually considered in stellarator optimization, such as the $\sqrt{\nu}$ and superbanana-plateau regimes. Optimization programmes are slowly starting to provide more a accurate characterization of transport by performing predictive simulations with prescribed sources and turbulent transport models. {\ttfamily KNOSOS} can also contribute to overcome two of the main limitations of this approach: the large computing time needed to create a database of mononoenergetic neoclassical transport coefficients and/or the lack of accuracy involved in the monoenergetic approach itself.
 
But a fast neoclassical code can have uses beyond stellarator optimization. For instance, the transport of impurities caused by their interaction with the bulk ions (via $\varphi_1$ or through inter-species collisions) has drawn much attention in the last years; however, a systematic study of its dependence on the magnetic configuration, colllisionality, and bulk plasma profiles remains to be done, due to the large computing resources needed for the combined solution of the quasineutrality and drift-kinetic equations of the bulk species. This will be addressed in forthcoming papers, in combination with analytical formulas for the radial flux of impurities in a variety of neoclassical regimes~\citep{calvo2018nf,calvo2019anis}.

Finally even in situations in which turbulence is dominant, a fast neoclassical code may be required. Its output (the radial electric field, the tangential electric field or the complete distribution function of the bulk species) can be read by gyrokinetic codes when studying the effect of neoclassical transport on turbulence. This effect is expected to be largest in those low-collisionality regimes in which the specificities of {\ttfamily KNOSOS} (very small computing time and inclusion of the tangential magnetic drift) are most relevant.
