\chapter{Motivation and scope}\label{CHAP_MOT}

\footnote{This chapter corresponds to the introduction of~\citep{velasco2019knosos}.}
Stellarators are non-axisymmetric devices in which the magnetic field is created basically by external magnets, without the need of any mechanism to drive current within the plasma. This provides them with an inherent capability for steady state operation and makes them less prone to plasma magnetohydrodynamic instabilities, but it also generally produces larger energy losses: at low collisionalities, the combination of magnetic geometry and particle collisions leads to a variety of stellarator-specific neoclassical transport regimes, which usually  give a large contribution to the radial energy and particle transport in the core of the device~\citep{dinklage2013ncval,dinklage2018np}. Of special relevance are the 1/$\nu$, the $\sqrt{\nu}$ and the superbanana-plateau regimes~\citep{hokulsrud1986neo,beidler2011icnts,calvo2017sqrtnu}, in which the energy transport coefficients show a positive temperature dependence, much more unfavourable than the negative $T^{-1/2}$ scaling of the banana regime of the axisymmetric tokamak. 

The fundamental reason for this behaviour is that in a generic stellarator, unlike in an axisymmetric tokamak, trapped particle orbits have non-zero secular radial drifts. The exception are omnigenous stellarators: in these magnetic configurations, the secular radial drifts vanish~\citep{cary1997omni,parra2015omni}, and the level of neoclassical transport is low, similar to that of the tokamak. Quasisymmetric stellarators~\citep{boozer1983qs} are a particular family of omnigenous stellarators, see e.g.~\citep{landreman2012omni}.

The two world's largest stellarators in operation, Wendelstein 7-X (W7-X)~\citep{klinger2017op11,wolf2017op11} and the Large Helical Device (LHD)~\citep{takeiri2017iaea}, have relied on optimization of neoclassical transport for their design and operation. The magnetic configuration of W7-X has been designed to be close to omnigeneity with poloidally-closed contours of the magnetic field strength; one of the goals of the project has been to prove the constructability and reliability of such designs~\citep{sunnpedersen2016nature}. In LHD, the plasma column can be shifted inwards so that the minimum values of the magnetic field along the field line have approximately the same value (see figure 2 of~\citep{beidler2011icnts}), a well-known geometric property of some omnigenous magnetic fields~\citep{mynick1982omni,landreman2012omni}; discharges performed using this magnetic configuration consistently show better energy confinement~\citep{yamada2005taue}. Finally, a particular kind of quasisymmetry, quasiaxisymmetry, was the design criterion of the National Compact Stellarator Experiment (NCSX)~\citep{zarnstorff2001ncsx}. Power reactor designs exist for these three stellarator concepts~\citep{sagara2010reactors}.

It is then clear that optimization of neoclassical transport is a crucial issue for a stellarator reactor. One of the most common goals of stellarator optimization efforts is the minimization of the so-called \textit{effective ripple}, a figure of merit that provides information of the level of transport in the 1/$\nu$ regime. While there is little doubt that minimization of this quantity should be a design criterion in any future stellarator, it has important limitations. On the one hand, empirical studies of the energy confinement time of several devices aimed at obtaining a unified International Stellarator Scaling law (ISS04) have not shown a very strong correlation between reduced effective ripple and improved energy confinement~\citep{yamada2005taue,fuchert2018taue}; on the other hand, self-consistent neoclassical transport simulations performed in the configuration space of W7-X, complemented with simplified anomalous modelling (accounting for non-negligible turbulent contributions to transport), have shown mild increases of the energy confinement time for configurations of significantly reduced effective ripple~\citep{geiger2014w7x}. This points towards one of the obvious limitations of the effective ripple: it is only an appropriate figure of merit for neoclassical transport if the plasma species are in the asymptotic 1/$\nu$ regime. However, bulk particles are distributed close to a Maxwellian that typically spans across several transport regimes. Even in cases in which the collisionality is low and the neoclassical predictions of the radial energy flux agree with the experiment, the parameter dependence of the experimental energy flux does not follow the scaling expected for any specific neoclassical transport regime, see e.g.~\citep{alonso2017eps}, because the flux is caused by a combination of transport regimes.
 
The reason for choosing the effective ripple as a figure of merit is that the 1/$\nu$ regime is the low-collisionality regime of stellarators in which the effect of the magnetic geometry on transport can be encapsulated in a straightforward manner in a single quantity that is independent of density, temperature and radial electric field. Furthermore, this quantity can be efficiently calculated by solving the bounce-averaged drift-kinetic equation, e.g. with the \texttt{NEO} code~\citep{nemov1999neo}. None of this has been possible so far for other low-collisionality regimes for arbitrary stellarator geometry.

Moreover, for other regimes such as the $\sqrt{\nu}$ and the superbanana-plateau regimes, the effect of the electric field (radial and tangential to the flux surface, the latter associated to the variation of the electrostatic potential on the flux surface, $\varphi_1$) has to be considered~\citep{calvo2017sqrtnu}, and this quantity is determined by imposing ambipolarity of the neoclassical particle fluxes and quasineutrality, which in turn depend on the plasma profiles, and specifically on the gradients. In order to address this issue, self-consistent neoclassical transport simulations have been performed in the last few years: the neoclassical fluxes are calculated with the~\DKES~code~\citep{hirshman1986dkes} and then the ambipolar and energy transport equations are solved (the latter with a prescribed energy source)~\citep{turkin2011predictive,geiger2014w7x}. Although we will see that~\DKES~makes use of the so-called monoenergetic approximation, which reduces the problem from five dimensions to three, using~\DKES~to self-consistently solve neoclassical energy transport is still computationally expensive at low collisionality. Moreover,~\DKES~is inaccurate at {sufficiently} low collisionality: it uses an incompressible $E\times B$ drift~\citep{beidler2007icnts} and does not include the tangential magnetic drift or the radial $E\!\times\!B$ drift caused by the variation of the electrostatic potential within the flux surface (the latter makes the fluxes depend non-linearly on the plasma gradients~\citep{calvo2018jpp}). Some or all of these approximations are absent in more recent codes such as \texttt{SFINCS}~\citep{landreman2014sfincs}, \texttt{EUTERPE}~\citep{regana2013euterpe,regana2017phi1} or \texttt{FORTEC-3D}~\citep{satake2006fortec3d}, but at the expense of higher computational cost.

We have developed a new code, the KiNetic Orbit-averaging Solver for Optimizing Stellarators, \texttt{KNOSOS}, based on the analytical techniques developed in a series of papers~\citep{calvo2013er,calvo2014er,calvo2015flowdamping,calvo2017sqrtnu,calvo2018jpp}. It solves local drift-kinetic equations that will be summarized in the next section and that accurately describe neoclassical transport in the $1/\nu$, $\sqrt{\nu}$ and superbanana-plateau regimes. The equations include the effect of the magnetic drift tangential to flux surfaces and the radial $E\!\times\!B$ drift due to the variation of the electrostratic potential within the flux surface; the radial electric field $E_r$ and $\varphi_1$ are obtained by imposing ambipolarity and quasineutrality, respectively. Local drift kinetic equations are valid for large-aspect-ratio stellarators or configurations close to omnigeneity (see the discussion before equation~(\ref{EQ_LOCAL}) in chapter \ref{CHAP_EQ}). Unlike preliminary versions of {\ttfamily KNOSOS}~\citep{velasco2018phi1,calvo2018jpp}, this version does not require a explicit split of the magnetic field magnitude into omnigeneous and non-omnigeneous pieces. The goal of this code is to be, at the same time, accurate and fast, so that it allows one to perform comprehensive parameter scans and to provide input to other codes or suites of codes. Generally speaking, the goal is to improve our confidence in neoclassical predictions, in light of recent theory developments, and to be able to fully exploit these predictive capabilities. To facilitate this objective, the code is freely-available and open-source.%\footnote{\todo{The source code and manual of {\ttfamily KNOSOS} can be downloaded from \href{https://github.com/joseluisvelasco/KNOSOS}{https://github.com/joseluisvelasco/KNOSOS}.}}.

