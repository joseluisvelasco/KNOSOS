\chapter{\DKES}\label{CHAP_DKES}

In this appendix, we describe how the code can read and use a database of transport coefficients calculated with \DKES. This can be done not only in order to compare them with the monoenergetic transport coefficients calculated (as in \S\ref{SEC_DKES}) but also to use them in combination with~\KNOSOS (as in \S\ref{SEC_TANGVM}). The reason is that the plateau approximation of \S\ref{SEC_PLATEAU} does not work for some devices (some examples of this and some discussion can be found in~\citep{beidler2011icnts}).

%%%%%%%%%%%%%%%%%%%%%%%%%%%%%%%%%%%%%%%%%%%%%%%%%%%%%%%%%%%%%%%%%%%%%%%%%%%%%%%%%%%%

\section{Input}\label{SEC_IDKES}

\KNOSOS~can read the output summary file {\ttfamily results.data} generated by \DKES. In order to be read, the transport coefficients must be distributed in folders and subfolders in a very precise manner. There should be {\ttfamily nefieldx} folders, named {\ttfamily omega\_?e??}, corresponding to  {\ttfamily nefieldx} values of {\ttfamily EFIELD}. Each of them should have  {\ttfamily ncmulx} subfolders, named {\ttfamily cmul\_?e??}, corresponding to  {\ttfamily ncmulx} values of {\ttfamily CMUL}. For the moment, the values are hardwired in {\ttfamily global.f90}, and they are:

\

{\ttfamily
\hskip-0.6cm INTEGER, PARAMETER :: ncmulx=18 \\
REAL*8 cmulx(ncmulx) /3E+2,1E+2,3E+1,1E+1,3E+0,1E+0,3E-1,\&\\ \&1E-1,3E-2,1E-2,3E-3,1E-3,3E-4,1E-4,3E-5,1E-5,3E-6,1E-6/\\
INTEGER, PARAMETER :: nefieldx=9\\
REAL*8 efieldx(nefieldx) /0E-0,1E-5,3E-5,1E-4,3E-4,1E-3,3E-3,1E-2,3E-2/}

\

\hskip-0.6cm and the names of the folders and subfolders are:

\

{\ttfamily
  \hskip-0.6cm dir\_efield(1) ='omega\_0e-0/'\\
  dir\_efield(2) ='omega\_1e-5/'\\
  dir\_efield(3) ='omega\_3e-5/'\\
  dir\_efield(4) ='omega\_1e-4/'\\
  dir\_efield(5) ='omega\_3e-4/'\\
  dir\_efield(6) ='omega\_1e-3/'\\
  dir\_efield(7) ='omega\_3e-3/'\\
  dir\_efield(8) ='omega\_1e-2/'\\
  dir\_efield(9) ='omega\_3e-2/'\\
  dir\_efield(10)='omega\_1e-1/'\\

\

  \hskip-0.6cm  dir\_cmul(1) ='cl\_3e+2/'\\
  dir\_cmul(2) ='cl\_1e+2/'\\
  dir\_cmul(3) ='cl\_3e+1/'\\
  dir\_cmul(4) ='cl\_1e+1/'\\
  dir\_cmul(5) ='cl\_3e-0/'\\
  dir\_cmul(6) ='cl\_1e-0/'\\
  dir\_cmul(7) ='cl\_3e-1/'\\
  dir\_cmul(8) ='cl\_1e-1/'\\
  dir\_cmul(9) ='cl\_3e-2/'\\
  dir\_cmul(10)='cl\_1e-2/'\\
  dir\_cmul(11)='cl\_3e-3/'\\
  dir\_cmul(12)='cl\_1e-3/'\\
  dir\_cmul(13)='cl\_3e-4/'\\
  dir\_cmul(14)='cl\_1e-4/'\\
  dir\_cmul(15)='cl\_3e-5/'\\
  dir\_cmul(16)='cl\_1e-5/'\\
  dir\_cmul(17)='cl\_3e-6/'\\
  dir\_cmul(18)='cl\_1e-6/'\\
}

It is not a problem if some extreme values of {\ttfamily EFIELD}~and of {\ttfamily CMUL} (specifically, the largest {\ttfamily EFIELD} and the smallest {\ttfamily CMUL}) are not available in the database (i.e., the database may contain e.g. 16 values of {\ttfamily CMUL}, but then they must exactly go from 3E+2 to 1E-5, and they must be distributed as shown above).

\

%%%%%%%%%%%%%%%%%%%%%%%%%%%%%%%%%%%%%%%%%%%%%%%%%%%%%%%%%%%%%%%%%%%%%%%%%%%%%%%%%%%%

\section{Use of \DKES~database}\label{SEC_UDKES}

The \DKES~database is used for one or several species if the corresponding \vlink{REGB} is -1 or 0. The database stores the logarithm of the monoenergetic transport coefficients (both the upper and the lower limit that \DKES~computes~\citep{hirshman1986dkes}) as a function of the logarithms of {\ttfamily CMUL} and {\ttfamily EFIELD}, and bidimensional Lagrange polynomial interpolation is done (one dimensional if  {\ttfamily EFIELD} is close enough to zero). %\todo{Two calculations of the fluxes are done}: one using the upper limit and the other using the lower limit: this provides an estimate of the precision of the calculation.

\

 if \vlink{REGB} is 0, \DKES~is used at high collisionalities \KNOSOS~as low collisionalities, as discussed in detail in \S\ref{SEC_TANGVM}). Alternatively, if \vlink{REGB} is -1, the calculation proceeds as follows:
\begin{itemize}
\item The usual interpolation of the monoenergetic database is done.
\item If \vlink{TANG\_VM} is set to~\true:
\begin{itemize}
\item a~\KNOSOS~monoenergetic calculation is done with \vlink{TANG\_VM} set to~\true~and \vlink{INC\_EXB} set to~\false;
\item a~\KNOSOS~monoenergetic calculation is done with \vlink{TANG\_VM} set to~\false~and \vlink{INC\_EXB} set to~\true;
\item the result of the latter is added to the interpolation, and the result of former is substracted.
\end{itemize}
\end{itemize}

\

No momentum conservation, as in~\cite{maassberg2009momentum}, is considered for the moment, since~\KNOSOS~does not calculate other mononergetic transport coefficientes needed, but its effect on radial transport should be negligible.


%%%%%%%%%%%%%%%%%%%%%%%%%%%%%%%%%%%%%%%%%%%%%%%%%%%%%%%%%%%%%%%%%%%%%%%%%%%%%%%%%%%%

\section{Normalization}\label{SEC_NORM}

\DKES~calculates the species-independent $\Gamma_{11}$ coefficient, which is related to $D_{11}$ via
\begin{equation}
\Gamma_{11} = -\frac{2}{v^3}\left(\frac{Z_be}{m_b}\right)^2\left(\frac{\partial\hat{r}}{\partial\psi}\right)^2 D_{11,b} \,,
\end{equation}\label{EQ_NORMDKES}
where $\hat{r}$ is the radial coordinate in \DKES. This yields
\begin{eqnarray}
\fsa{\mathbf{\Gamma}_b\cdot \bnabla\psi} &=& \pi \left(\frac{m_b}{Z_be}\right)^2\left(\frac{\partial\psi}{\partial\hat{r}}\right)^2 \int_0^\infty\mathrm{d}v^2\,v^4 f_{M,b} \Gamma_{11}\Upsilon_b =\nonumber\\ 
&=& 2\pi \left(\frac{m_b}{Z_be}\right)^2\left(\frac{\partial\psi}{\partial\hat{r}}\right)^2 \int_0^\infty\mathrm{d}v\,v^5 f_{M,b} \Gamma_{11}\Upsilon_b\,.
\label{EQ_FLUX2}
\end{eqnarray}
%This way, one recovers~\citep{beidler2011icnts} 
%\begin{eqnarray}
%\fsa{\mathbf{\Gamma}_b\cdot \bnabla\psi} = - n_b\frac{2}{\sqrt{\pi}}\int_0^\infty\mathrm{d}K_b\,K_b^{1/2}\exp{(-K_b)} D_{11}\Upsilon\,,
%\end{eqnarray}
%where $K_b=\frac{m_bv^2}{2T_b}$ which, using the expression for the Maxwellian distribution, can be rewritten
%\begin{eqnarray}
%\fsa{\mathbf{\Gamma}\cdot \bnabla\psi} &=& - 2\pi\int_0^\infty\mathrm{d}v^2\,v f_M D_{11}\Upsilon = \nonumber\\ &=& - 4\pi\int_0^\infty\mathrm{d}v\,v^2 f_M D_{11}\Upsilon\,.\label{EQ_FLUX}
%\end{eqnarray}

Since, when benchmarking with \DKES, we will compare monoenergetic simulations for several values of the parameters {\ttfamily CMUL}$=\nu/v$ and {\ttfamily EFIELD}$=E_{\hat{r}}/v$, we need to keep in mind that
\begin{eqnarray}
E_{\hat{r}}= -\frac{\partial\varphi_0}{\partial\hat{r}} = -\frac{\partial\varphi_0}{\partial\psi}\frac{\partial\psi}{\partial\hat{r}}=E_\psi\frac{\partial\psi}{\partial\hat{r}}\,,
\end{eqnarray}
and also (see chapter~(\ref{CHAP_EQ}))
\begin{eqnarray}
\nu = \nu_\lambda/2\,.
\end{eqnarray}

%\
%
%Note also that, since $p=\sigma\sqrt{1-\lambda B}$:
%\begin{equation}
%\int_{-1}^{+1}\mathrm{d}p = \int_{1/B_{min}}^{1/B_{max}}\mathrm{d}\lambda\frac{B}{\sqrt{1-\lambda B}}\,,
%\end{equation}
